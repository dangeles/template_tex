\documentclass[9pt,twocolumn,twoside]{rvw}
\articletype{inv} % article type

\usepackage[colorinlistoftodos]{todonotes}
\usepackage{gensymb}
\usepackage{siunitx}
\usepackage{epigraph}

\newcommand{\ra}[1]{\renewcommand{\arraystretch}{#1}}

\title{A smart title}

\author[1]{David Angeles-Albores}
\author[1,2,3]{Matt Thomson}
\affil[1]{Biological Engineering, Massachusetts Institute of Technology,
Cambridge, MA}
\affil[2]{Center for Microbiome, Informatics and Therapeutics, Massachusetts
Institute of Technology, Cambridge, MA}
\affil[3]{Broad Institute, Cambridge, MA}

\keywords{keyword 1; keyword 2}
\runningtitle{running title} % For use in the footer
\correspondingauthor{Angeles-Albores}

\begin{abstract}
  TBD
\end{abstract}
\setboolean{displaycopyright}{true}


% document begins here
\begin{document}

\maketitle{}
\thispagestyle{firststyle}
% \marginmark{}
\firstpagefootnote{}
\correspondingauthoraffiliation{E-mail: ejalm@mit.edu
        }
\vspace{-11pt}%
\linenumbers{}

% I don't usually have an introduction heading.
\lettrine[lines=2]{\color{color2}D}{ocument} starts Here

\section{Methods}

\section{Results}

\section{Conclusions}








%This is where your bibliography is generated.
% \bibliography{citations}
% \bibliographystyle{genetics}

\end{document}
